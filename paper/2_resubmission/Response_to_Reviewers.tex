\documentclass[11pt, a4paper, answers]{exam}

\usepackage{shadow, fancybox, a4wide, bm, harvard, epsfig, theorem,  epic, verbatim, amsmath, amssymb,  microtype, fullpage, setspace, caption, makeidx, subfigure, marvosym, booktabs, dcolumn, mdwlist, lastpage, tikz, eurosym}
\usepackage[OT1]{fontenc}
\usepackage[utf8]{inputenx}
\renewcommand{\baselinestretch}{1.25}


%\hypersetup{%
%	colorlinks,urlcolor=blue,
%	pdfborder = {0 0 0}
%}

\usepackage{xcolor}
%\definecolor{SolutionColor}{rgb}{0.1,0.3,1}
%\renewcommand{\thequestion}{\alph{question} }
\renewcommand\questionlabel{\llap{\thequestion)}}

\renewcommand{\solutiontitle}{\noindent\textbf{Response: }}


%\pointsinrightmargin
%\boxedpoints
%\unframedsolutions
\shadedsolutions
\definecolor{SolutionColor}{rgb}{0.9,0.9,1}

\begin{document}


\noindent Dear Editor and Reviewers, \bigskip\bigskip

Please find attached a report covering our detailed responses to the editor and 
the reviewer comments. Overall, we feel the paper has been substantially improved 
as a result of the constructive criticism and suggestions made in the  review process,
and we would like to sincerely thank both reviewers as well as the editor for
that. We have structured this response by listing every
comment and immediately following it with a blue box that includes our
response.

%\today
\clearpage

\section*{Responses to Editor}

\begin{questions}
\question The second paragraph in the literature is too long. Consider breaking 
it into two or three paragraphs based on reclassification of the literature, for example, 
by substitution and complementarity effects, or by urban and rural areas.

\begin{solution}
\textbf{ET}
\end{solution}

\question For clarity, I would like to suggest you replace Zipf coefficient by Pareto 
exponent unless the coefficient equals 1, as you rightfully pointed it out in the 
paragraph below equation (1). This will help readers avoid some confusions.

\begin{solution}
\textbf{YMI}
\end{solution}

\question Please clarify in equation (3) whether you want to include time trend 
delta*t or year fixed effect delta subscript t. You used delta*t in the equation 
but used delta subscript t in line 234. Tables 4 and 5 say year fixed effects.

\begin{solution}
\textbf{ET}
\end{solution}

\question Using female labor force participation rate as IV for ICT usage is a bit 
exotic. Even females stay home as housewives, don’t they still need to use phones 
and internet at home? You did not provide any empirical evidence that female labor 
force participation indeed increases ICT usage. It would be more convincing if 
you can find and cite such supportive references.  

\begin{solution}
\textbf{ET}
\end{solution}

\question Table 5: the impact on Zipf coefficient is very small: 0.001 or 0.002. 
How economically significant is this number in terms of city size distribution? 
Particularly, since a country uses both internet and fixed line phones and they 
have opposite, offsetting effects on Zipf coefficient, does this mean at the 
aggregate level there is no effect on the overall city size distribution? 
These are subtle questions and you may discuss a bit on this.

\begin{solution}
\textbf{YMI}
\end{solution}

\question Line 504: you mentioned using absolute number of broadband tests 
in 2011 as an additional IV, isn’t it a direct measure of ICT and similar 
to the main explanatory variable “broadband tests per capita 2011”?

\begin{solution}
\textbf{ET}
\end{solution}

\end{questions}

\section*{Responses to Reviewer 1}

\begin{questions}

\question There are two kinds of spatial dispersion: within a city or 
across different cities. They are conceptually different; there could be 
decreased dispersion within a city but at the same time increased 
dispersion across cities. Both types of spatial dispersion can be 
affected by the advancement of information technology. This study 
examines the “Zipf coefficient” of an urban system or a city’s rank 
in an urban system, thus it focuses on spatial dispersion across cities. 
Nothing is said about spatial dispersion within a city (e.g., whether 
ICT has resulted in more urban sprawl). The authors should make this 
clear upfront.

\begin{solution}
\textbf{YMI: maybe add a line or two in the intro where we talk about data?}
\end{solution}

\question After equation (1), it’ll be useful to add a brief discussion 
on why the Zipf coefficient measures spatial dispersion. Many 
readers should have seen the variance, Gini coefficient, or 
Herfindahl index used as a measure of dispersion. I am not 
sure how many of them has seen the Zipf coefficient, so it 
requires some explanation. I also wonder why those other 
measures are not used.

\begin{solution}
\textbf{YMI for the text, ET to check if we have estimates for 
alternative measures}
\end{solution}

\question In Table 2, the Zipf coefficient ranges between -2 and -1, 
but has a mean of -1 and a standard deviation of 0. That seems odd. 
Also, what is the “St. error for Zipf est.” in the second row?

\begin{solution}
\textbf{ET}
\end{solution}

\question On several occasions (e.g., p. 9 and 15), the authors 
seem to suggest that a valid IV should not affect (or be correlated 
with) the outcome variable. This is not entirely accurate. A valid 
IV must be correlated with the endogenous variable. Thus, if the 
endogenous variable indeed has a causal effect on the outcome variable, 
the IV must be correlated with the outcome variable. It would be 
correct to say that an IV would not affect the outcome variable 
directly and would only affect it indirectly (i.e., through the 
endogenous variable). However, no direct effect is different 
from a zero correlation between the IV and the outcome variable. 
Thus showing a low correlation between the IV and the outcome 
variable (p. 12, 15, and 18) is not informative about the validity of the IV.

\begin{solution}
\textbf{ET will remove correlations}
\end{solution}

\question In the global study, the use of an IV has reversed the 
sign of an OLS estimate and made two statistically insignificant 
OLS estimates become significant. In both case studies, the use 
of an IV has produced much larger estimates than the OLS 
estimates. If the authors really trust the IV estimates, it will be 
helpful to discuss what have biased the OLS estimates in the 
particular direction. For example, in both case studies, one 
might suspect that omitted variables could bias the OLS 
estimates upward. However, surprisingly the IV estimates 
turned out to be even larger than the OLS estimates. How do 
we understand this?

\begin{solution}
\textbf{ET will give first it a go then YMI }
\end{solution}

\question When using an instrumental variable, it will be 
helpful to offer some intuition why the IV is correlated with 
the endogenous variable. For example, why does female 
participation in labor force affect internet and phone usage? 
Why does the absolute number of broadband tests in 2011 
affect the endogenous variable (Internet speed)? These are 
not obvious, so some discussion is needed. Also, it might be 
helpful to show (or simply mention) some of the first-stage 
regression results to let readers know how the IV and the 
endogenous variable are correlated. For example, I would 
like to know whether and to what extent the absolute number 
of broadband tests in 2011 increased Internet speed. This 
information helps readers gauge whether the overall IV 
strategy makes sense.

\begin{solution}
\textbf{ET}
\end{solution}

\question In the two case studies, I am not sure that the 
normalization of the changed rank is helpful. I would prefer 
to use the unnormalized change in rank as the left-hand 
side variable for the OLS regression, because it makes the 
key coefficient easier to interpret. Normalization allows for 
the estimation of the quasibinomial GLM, but the GLM results 
are never discussed. It might make sense to use the unnormalized 
change in rank as the left-hand side variable, keep the OLS results 
only, and simply mention the GLM exercise in a footnote.

\begin{solution}
\textbf{ET}
\end{solution}

\end{questions}

\section*{Responses to Reviewer 2}

\begin{questions}

\question The authors used different instrumental variables for 
the multi-county, US, and UK analyses. Why is female labor 
participation not used as a consistent IV throughout the 
analyses? Is it because of different contexts or limited variation? 
The US and UK analyses also use similar yet different IV's. 
Perhaps the authors can further discuss their rationale in changing choices of IV's.

\begin{solution}
\textbf{ET}
\end{solution}

\question In interpreting the empirical results, the authors, 
to some extent, highlight the potential mechanism in explaining 
the positive relationship between internet adoption and urban 
structure. However, there is limited discussion on other interesting 
findings. For instance, why does fixed telephony have an opposite 
effect to internet and digital communication technologies?

\begin{solution}
\textbf{YMI, maybe an argument based on time/maturity of technology/link to the 
2008 paper?}
\end{solution}

\question Visualizing the patterns in Table 1 in the format of maps 
may make the information more reader friendly.

\begin{solution}
\textbf{ET}
\end{solution}

\question There are minor formatting issues and typos throughout 
the paper. For examples, "Government expenditure ( Trade (\% of GDP)" 
in Table 4 and "cites" in line 343.

\begin{solution}
\textbf{ET}
\end{solution}
\end{questions}

\clearpage
\end{document}
